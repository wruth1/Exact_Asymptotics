\documentclass{article}

\usepackage{amsmath, amssymb}
\usepackage{xcolor}
\usepackage{caption, subcaption}
\usepackage{graphicx}
\usepackage{bm}             % For bold math symbols

\usepackage{natbib}
\bibliographystyle{plainnat}


\newcommand{\bG}{\mathbb{G}}
\newcommand{\bF}{\mathbb{F}}
\newcommand{\bP}{\mathbb{P}}
\newcommand{\bE}{\mathbb{E}}
\newcommand{\bV}{\mathbb{V}}

\newcommand{\zdens}{\frac{e^{-z^2/2}}{\sqrt{2\pi}}}


\title{Verification of SE Formulas for Mediation Analysis}

\begin{document}

\maketitle

There is a short exchange in the American Journal of Epidemiology

\section{The \citeauthor{Sam23} group}

This group does essentially what B\&B are proposing with exact formulas and $\delta$-method standard errors. In particular, they avoid any requirement for rare events. They also do direct and indirect effects. However, I haven't seen any mention of random-effects/multilevel-models.


\subsection{\citet{Sam18}}

Gives exact formulas for direct and indirect mediation effects, as well as $\delta$-method standard errors. Binary outcome, binary mediator. No analytical SE when exposure-mediator interaction is present.

\subsection{\citet{Sam21}}

Extends analytical results of \citet{Sam18} to handle exposure-mediator interaction term in model for outcome.

\subsection{\citet{Sam23}}

As \citet{Sam18}, but with continuous mediator and binary response. No mixed-effects. Very nice simulation study.

\subsection{\citet{Cau24}}

Extends the work of \citet{Sam23} to case-control data.

\section{Derivative-Based}

Defines mediation effects in terms of derivatives. E.g. They define the indirect effect of $X$ on $Y$ as $(\partial \bE Y/ \partial \bE M) \cdot (\partial \bE M / \partial X)$, although that's not quite how they state it in the paper. This approach has had a small, but consistent following.

\subsection{\citet{Mar24}}

Incorporates random-effects in the derivative-based approach. Mostly does Bayesian inference. Mentions bootstrap.


\section{GLMM UQ}

\subsection{\citet{Zhe21}}

Does joint uncertainty quantification for the parameters and random effects. Based on maximizing the Laplace Approximation to the marginal likelihood. Gives $\delta$-method SEs with gradients from automatic differentiation 


\section{Applications}

\subsection{Not Relevant}

\citet{Bro22} talks about mediation analysis and includes models with mixed-effects, but the latter are only used for imputation.




\bibliography{Lit_Review_Bib}

\end{document}