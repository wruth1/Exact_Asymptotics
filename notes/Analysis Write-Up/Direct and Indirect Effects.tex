\documentclass{article}

\usepackage{amsmath, amssymb}
\usepackage{xcolor}
\usepackage{caption, subcaption}
\usepackage{graphicx}
\usepackage{bm}             % For bold math symbols
\usepackage{mathrsfs}       % For the \mathscr{} font

\usepackage{natbib}
\bibliographystyle{plainnat}

\usepackage{hyperref}   % For the \url{} command

\usepackage{tabu,multirow}



\newcommand{\bG}{\mathbb{G}}
\newcommand{\bF}{\mathbb{F}}
\newcommand{\bP}{\mathbb{P}}
\newcommand{\bE}{\mathbb{E}}
\newcommand{\bV}{\mathbb{V}}

\newcommand{\sY}{\mathcal{Y}}

\newcommand{\zdens}{\frac{e^{-z^2/2}}{\sqrt{2\pi}}}


\title{Analysis of Direct and Indirect Mediation Effects in Causal Mediation Analysis}

\begin{document}

\maketitle

In this document, we develop definitions and formulas for the direct and indirect mediation effects to complement the total effect given by B \& B. We follow the work of \citet{Pea12} and the Samoilenko and Lefebvre group (e.g., \citealp{Sam23}). 

\section{Expected Nested Counterfactuals}
\label{sec:nest_CFs}

To start, we define the nested counterfactual $Y(x, M(x'))$ as the value that $Y$ would assume when $X$ is set to $x$ and $M$ is set to whatever value it would have assumed if we had set $X$ to $x'$. \citet{Pea12} identifies the expected value of this nested counterfactual with the following expression based on conditional expectations:
%
\begin{align}
    \bE Y(x, M(x')) = \int \bE(Y | M=m, X=x) \bP(M = dm | X = x') \label{eq:med_gen}
\end{align}
%
Focusing now on the case with binary response and binary mediator, \ref{eq:med_gen} becomes
%
\begin{multline}
    \bE Y(x, M(x')) = \bP(Y=1 | M=1, X=x) \bP(M=1 | X=x') + \\\bP(Y=1 | M=0, X=x) \bP(M=0 | X=x') \label{eq:med_bin}
\end{multline}
%
Consider using  logistic regression models for $Y$ and $M$. Write $\mathrm{logit}(\bE(Y | M=m, X=x)) = \beta_0 + \beta_m m + \beta_x x$ and $\mathrm{logit}(\bE(M | X=x)) = \alpha_0 + \alpha_x x$. Then (\ref{eq:med_bin}) becomes
%
\begin{multline}
    \bE Y(x, M(x')) = \frac{1}{1 + \exp(-\beta_0 - \beta_m - \beta_x x)} \frac{1}{1 + \exp(-\alpha_0 - \alpha_x x')} +\\
    \frac{1}{1 + \exp(-\beta_0 - \beta_x x)} \frac{1}{1 + \exp(\alpha_0 + \alpha_x x')} \label{eq:med_bin_fix}
\end{multline}
%
Equation (\ref{eq:med_bin_fix}) holds for logistic regression with fixed-effects only. If we instead use  mixed-effects logistic regressions for $Y$ and $M$, then (\ref{eq:med_gen}) and (\ref{eq:med_bin}) still hold. For the mixed-effects models, first, write $V = (V_0, V_m, V_x) \sim \mathrm{N}(0, \Sigma_V)$ and $U = (U_0, U_x) \sim \mathrm{N}(0, \Sigma_U)$ for the random-effects in our models for $Y$ and $M$ respectively. Next, write $\mathrm{logit}(\bE(Y | V, M=m, X=x)) = \beta_0 + V_0 + (\beta_m + V_m) m + (\beta_x + V_x) x$ and $\mathrm{logit}(\bE(M |U, X=x)) = (\alpha_0 + U_0) + (\alpha_x + U_x) x$. Returning now to identification of the expected counterfactual for $Y$, we get
%
\begin{align}
    \bE Y(x, M(x')) =& \left[\int \bP(Y=1 |V=v, M=1, X=x) \bP(V = dv)\right] \cdot \\
    & \left[\int \bP(M=1 |U=u, X=x') \bP(U=du)\right] +\\
     & \left[\int \bP(Y=1 |V=v, M=0, X=x) \bP(V = dv)\right] \cdot \\
     & \left[\int \bP(M=0 |U=u, X=x') \bP(U=du)\right]
\end{align}
%
and, in the logistic regression context,
%
\begin{align}
    \bE Y(x, M(x')) =& \left[ \int \frac{\phi(v; 0, \Sigma_V)}{1 + \exp(-(\beta_0 + v_0) - (\beta_m + v_m) - (\beta_x + v_x) x)}  dv \right] \cdot \label{eq:med_bin_ran}\\
    &\left[ \int \frac{\phi(u; 0, \Sigma_U)}{1 + \exp(-(\alpha_0 + u_0) - (\alpha_x + u_x) x')}  du \right] + \nonumber\\
    & \left[\int \frac{\phi(v; 0, \Sigma_V)}{1 + \exp(-(\beta_0 + v_0) - (\beta_x + v_x) x)} dv \right] \cdot \nonumber\\
    & \left[\int \frac{\phi(u; 0, \Sigma_U)}{1 + \exp((\alpha_0 + u_0) + (\alpha_x + u_x) x')} du \right], \nonumber
\end{align}
%
where $\phi(.; \mu, \Sigma)$ is the multivariate normal density with mean $\mu$ and covariance matrix $\Sigma$. 

Note that the four integrals in (\ref{eq:med_bin_ran}) are all multivariate, but can be transformed to univariate integrals by suitable changes of variables. To this end, write $\eta = \alpha_0 + \alpha_x x$ and $\zeta = \beta_0 + \beta_x x$ for two linear predictors (note that $\zeta$ does not contain $\beta_m$), and $\gamma^2_\Sigma(c_1, \ldots, c_r) = (c_1, \ldots, c_r) \Sigma (c_1, \ldots, c_r)^T$, where $\Sigma$ is an $r$-by-$r$ covariance matrix. We will generally set $\Sigma = \Sigma_V$ or $\Sigma = \Sigma_U$, in which case we write $\gamma^2_V$ or $\gamma^2_U$ respectively. We now define the function $\psi$ as a template for the four integrals in (\ref{eq:med_bin_ran}). 
%
$$
\psi(\mu, \sigma^2) := \int \frac{\phi(z; 0, 1)}{1 + \exp(-\mu - \sigma z)} dz.
$$
%
Note that $\psi$ is a univariate integral, so we can expect it to be well-approximated by routine numerical quadrature techniques. We now re-write (\ref{eq:med_bin_ran}) in terms of $\psi$ as follows,
%
\begin{align}
    \bE Y(x, M(x')) =& \psi(\zeta + \beta_m, \gamma^2_V(1, 1, x)) \cdot \psi(\eta, \gamma^2_U(1, x)) +\\
    & \psi(\zeta, \gamma^2_V(1, 0, x)) \cdot \psi(-\eta, \gamma^2_U(1, x)). \nonumber 
\end{align}


\section{Mediation Effects}
Denote the expected nested counterfactual defined in (\ref{eq:med_gen}) by $\mathscr{Y}(x, x') = \bE Y(x, M(x'))$. We can define the various mediation effects in terms of expected counterfactuals. Note that mediation effects for a binary outcome are commonly defined on three different scales: risk difference, risk ratio and odds ratio. Table \ref{tab:med_eff_defs} gives all such definitions explicitly. Combined with the formulas given throughout Section \ref{sec:nest_CFs}, we now have everything we need to do point estimation for the various mediation effects. It still remains however, to address uncertainty quantification.
%
\begin{table}[ht]
    \centering
    \caption{Definitions of various mediation effects; $x$ and $x'$ denote different values of the exposure.}
    \label{tab:med_eff_defs}
    \begin{tabu}{|[1pt]c|c|c|[1pt]}
        \tabucline[1pt]{-}
        \multirow{3}{*}{Risk Difference} & Total Effect & $\sY(x, x) - \sY(x', x')$ \\
        \cline{2-3}
        & Direct Effect & $\sY(x, x') - \sY(x', x')$\\
        \cline{2-3}
        & Indirect Effect & $\sY(x, x) - \sY(x, x')$ \\
        \tabucline[1pt]{-}
        \multirow{3}{*}{Risk Ratio} & Total Effect & $\sY(x, x) / \sY(x', x')$ \\
        \cline{2-3}
        & Direct Effect & $\sY(x, x') / \sY(x', x')$\\
        \cline{2-3}
        & Indirect Effect & $\sY(x, x) / \sY(x, x')$ \\
        \tabucline[1pt]{-}
        \multirow{3}{*}{Odds Ratio} & Total Effect & $\left. \frac{\sY(x, x)}{1 - \sY(x, x)} \right/ \frac{\sY(x', x')}{1 - \sY(x', x')} $\\
        \cline{2-3}
        & Direct Effect & $\left. \frac{\sY(x, x')}{1 - \sY(x, x')} \right/ \frac{\sY(x', x')}{1 - \sY(x', x')} $\\
        \cline{2-3}
        & Indirect Effect & $\left. \frac{\sY(x, x)}{1 - \sY(x, x)} \right/ \frac{\sY(x, x')}{1 - \sY(x, x')} $\\
        \tabucline[1pt]{-}
    \end{tabu}
\end{table}

\subsection{Uncertainty Quantification}

Note that all mediation effects are defined in terms of expected nested counterfactuals $\sY$. Thus, if we can produce a formula for the asymptotic covariance matrix of two (or more) values of $\sY$, we can use a simple application of the $\delta$-method to obtain the variance of any mediation effect. In fact, we can obtain the covariance matrix for all three mediation effects defined on a particular scale (e.g., risk ratio), or, indeed, between all nine mediation effects given in Table \ref{tab:med_eff_defs}. 

Toward a variance formula for $\sY$, we first write $\theta$ for all parameters upon which $\sY$ depends. That is, $\theta$ contains both sets of regression coefficients, $\beta_0, \beta_m, \beta_x$ and $\alpha_0, \alpha_x$, as well as both sets of covariance parameters. For consistency with B \& B, we parameterize these as $\tau_0, \tau_m, \tau_x$ for the standard deviations of $V_0, V_m, V_x$, and $\tau_{0,m}, \tau_{0,x}, \tau_{m,x}$ for the corresponding correlations\footnote{While our notation doesn't match that given by B \& B, parameterizing in terms of the standard deviations and correlations does. Alternative choices include the variances and covariances, or the unique components of the Cholesky factorizations of $\Sigma_V$ and $\Sigma_U$.}. Similarly, we use $\sigma_0, \sigma_x$ for the standard deviations of $U_0, U_x$, and $\sigma_{0,x}$ for their correlation. We are now equipped to write-out $\theta$ in full. The order of parameters is chosen to match my code (and to avoid refactoring thereof).
$$
\theta = (\alpha_0, \alpha_x, \sigma_0, \sigma_x, \sigma_{0,x}, \beta_0, \beta_m, \beta_x, \tau_0, \tau_m, \tau_x, \tau_{0,m}, \tau_{0,x}, \tau_{m,x})
$$
%
There are two problems left to solve for the quantification of uncertainty in $\sY$. First, we need the (asymptotic) covariance matrix of $\theta$. Second, we need the $\theta$-gradient of the function $\sY$. Note that $\sY$ is also a function of $x$ and $x'$, but we treat these values as fixed for our analysis.

\section{To Do}
\begin{itemize}
    \item Incorporate an interaction term between $X$ and $M$ in the model for $Y$.
\end{itemize}

\bibliography{../Literature_Review/Lit_Review_Bib.bib}
% \bibliography{Lit_Review_Bib}
% \bibliography{notes/Literature Review/Lit_Review_Bib}

\end{document}